\chapter{Fundamentação Teórica}
	
\section{Criptografia}
	
A palavra criptografia tem origem no grego cryptos, que significa segredo, oculto, e isto já nos dá uma boa ideia do que significa criptografar. Com o intuito de que uma mensagem transmitida, seja ela um texto, imagem, áudio ou qualquer outro de informação seja entendida apenas pelo destinatário desejado, podemos criptografá-la, ou seja, tornar ilegível a qualquer um que não contenha a chave, necessária que a mensagem seja decriptografada. A criptografia pode ser então entendida como uma ferramenta de segurança, garantindo que apenas quem ou o que possuir autorização possa interpretar a mensagem enviada.\par
Conforme a tecnologia avança se torna mais fácil e rápido o processamento de dados, permitindo a evolução dos algoritmos de criptografia, porém evolui também os métodos e algoritmos para quebrar e decifrar mensagens criptografadas, exigindo maiores níveis de segurança de informação. 

	
	\subsection{Criptografia Simétrica }
	
Criptografia simétrica é o tipo de criptografia que utiliza a mesma chave, tanto para o processo de criptografia quanto para a decriptografia.

\subsection{Criptografia Assimétrica }
A criptografia assimétrica utiliza chaves diferentes para o processo de criptografia e decriptografia
\section{Bit x Byte x Word}

Um bit não é nada além de um binário, ou seja pode assumir o valor 0 ou 1. Um byte é um conjunto de 8 bits. Uma word representa um conjunto de bytes, não apresentando tamanho fixo, mas uma multiplicação de um fato inteiro pelo número de bytes.
\section{Operação XOR}

Esta operação nada mais é do que uma comparação. Comparando 2 bits, se estes forem iguais a operação retorna o bit 0 se estes forem diferentes a operação retorna o bit 1.

	
	
	\section{AES (Advanced Encryption Standard)}

	É uma cifra, para criptografia simétrica, que utiliza tamanho de bloco de 128 bits e o tamanho de chave pode variar entre 128,192 ou 256 bits. Isto significa que a entrada, mensagem, será um bloco de 128 bits, o tamanho da chave será definido entre as opções ditas acima e o algoritmo responsável pela criptografia, retornará um bloco de 128 bits, chamado cifra. Utilizando a chave correta, e como entrada a cifra de 128 bits, retornaremos a mensagem original.
	
		\begin{table}[tbh]
		\centering
		\caption{Parâmtros do AES}
		\label{tab1}
		\begin{tabular}{|c|c|c|c|}
			\hline
			\multicolumn{1}{|l|}{\textbf{Tamanho da Chave (words/bytes/bits)}} & 4/16/128 & 6/24/192 & 8/32/256 \\ \hline
			\textbf{Tamanho do Bloco de entrada (words/bytes/bits)}            & 4/16/128 & 4/16/128 & 4/16/128 \\ \hline
			\textbf{Número de Rodadas}                                         & 10       & 12       & 14       \\ \hline
			\textbf{Tamanho da Chave da Rodada (words/bytes/bits)}             & 4/16/128 & 4/16/128 & 4/16/128 \\ \hline
			\textbf{Tamanho da Chave Expandida (words/bytes)}                  & 44/176   & 52/208   & 60/240   \\ \hline
		\end{tabular}
	\end{table}
	
	
	\subsection{Algoritmo de Rijndael}
	
	O algoritmo apresenta um número limitado de rodadas para concluir o processo de criptografia, este número depende do tamanho de chave escolhido, sendo assim:
	
		\begin{itemize}
		\item 10 se o bloco e a chave forem de 128 bits; 
		\item 12 se o bloco ou a chave forem de 192 bits, e nenhum deles for maior que isso; 
		\item 14 se o bloco ou a chave forem de 256 bits. 
		
		\end{itemize}
	
	Observando que para entender os requisitos do AES o bloco utilizado será sempre de 128 bits. Este bloco é representado por uma matriz quadrada de bytes e copiado para um vetor State, este vetor é alterado em cada etapa, seja na criptografia ou decriptografia. \par
	Para cada rodada então teremos quatro estágios diferentes, sendo um de permutação e três de substituição. Os quatro estágios e suas funções são os seguintes:
	\begin{description}
		\item[SubBytes:] Utiliza uma matriz, caixa-S, para realizar a substituição byte a byte do bloco
		\item[ShiftRows:] Uma permutação simples
		\item[MixColumns:] Uma combinação linear
		\item[AddRoundKey:] Um XOR bit a bit
	\end{description}
	 
		Tanto para a criptografia, quanto para a decriptografia, a cifra inicia no estágio AddRoundKey, seguido de (n-1) rodadas com os quatro estágios, sendo n igual a o número total de rodadas, e uma última rodada apenas com três estágios. É importante notar que apenas o estágio AddRoundKey utiliza a chave, sendo o único que agrega segurança. Os demais estágios são importantes para adicionar confusão, difusão e não-linearidade.
	
	\subsubsection{Estágio SubBytes}
Este estágio consiste numa substituição direta de bytes. O AES definiu uma matriz, chamada Caixa-S, com 16x16 elementos, de valores de bytes. Esta matriz contém todos os valores possíveis de 8 bits permutados.\par
\begin{table}[tbh]
	\centering
	\caption{Caixa-S}
	\label{tab2}
	\begin{tabular}{c|c|c|c|c|c|c|c|c|c|c|c|c|c|c|c|c|}
		\cline{2-17}
		& \textbf{0} & \textbf{1} & \textbf{2} & \textbf{3} & \textbf{4} & \textbf{5} & \textbf{6} & \textbf{7} & \textbf{8} & \textbf{9} & \textbf{A} & \textbf{B} & \textbf{C} & \textbf{D} & \textbf{E} & \textbf{F} \\ \hline
		\multicolumn{1}{|c|}{\textbf{0}} & 63         & 7C         & 77         & 7B         & F2         & 6B         & 6F         & C5         & 30         & 01         & 67         & 2B         & FE         & D7         & AB         & 76         \\ \hline
		\multicolumn{1}{|c|}{\textbf{1}} & CA         & 82         & C9         & 7D         & FA         & 59         & 47         & F0         & AD         & D4         & A2         & AF         & 9C         & A4         & 72         & C0         \\ \hline
		\multicolumn{1}{|c|}{\textbf{2}} & B7         & FD         & 93         & 26         & 36         & 3F         & F7         & CC         & 34         & A5         & E5         & F1         & 71         & D8         & 31         & 15         \\ \hline
		\multicolumn{1}{|c|}{\textbf{3}} & 04         & C7         & 23         & C          & 18         & 96         & 05         & 9A         & 07         & 12         & 80         & E2         & EB         & 27         & B2         & 75         \\ \hline
		\multicolumn{1}{|c|}{\textbf{4}} & 09         & 83         & 2C         & 1A         & 1B         & 6E         & 5A         & A0         & 52         & 3B         & D6         & B3         & 29         & E3         & 2F         & 84         \\ \hline
		\multicolumn{1}{|c|}{\textbf{5}} & 53         & D1         & 00         & ED         & 20         & FC         & B1         & 5B         & 6A         & CB         & BE         & 39         & 4A         & 4C         & 58         & CF         \\ \hline
		\multicolumn{1}{|c|}{\textbf{6}} & D0         & EF         & AA         & FB         & 43         & 4D         & 33         & 85         & 45         & F9         & 02         & 7F         & 50         & 3C         & 9F         & A8         \\ \hline
		\multicolumn{1}{|c|}{\textbf{7}} & 51         & A3         & 40         & 8F         & 92         & 9D         & 38         & F5         & BC         & B6         & DA         & 21         & 10         & FF         & F3         & D2         \\ \hline
		\multicolumn{1}{|c|}{\textbf{8}} & CD         & 0C         & 13         & EC         & 5F         & 97         & 44         & 17         & C4         & A7         & 7E         & 3D         & 64         & 5D         & 19         & 73         \\ \hline
		\multicolumn{1}{|c|}{\textbf{9}} & 60         & 81         & 4F         & DC         & 22         & 2A         & 90         & 88         & 46         & EE         & B8         & 14         & DE         & 5E         & 0B         & DB         \\ \hline
		\multicolumn{1}{|c|}{\textbf{A}} & E0         & 32         & 3A         & 0A         & 44         & 06         & 24         & 5C         & C2         & D3         & AC         & 62         & 91         & 95         & E4         & 79         \\ \hline
		\multicolumn{1}{|c|}{\textbf{B}} & E7         & C8         & 37         & 6D         & 8D         & D5         & 4E         & A9         & 6C         & 56         & F4         & EA         & 65         & 7A         & AE         & 08         \\ \hline
		\multicolumn{1}{|c|}{\textbf{C}} & BA         & 78         & 25         & 2E         & 1C         & A6         & B4         & C6         & E8         & DD         & 74         & 1F         & 4B         & BD         & 8B         & 8A         \\ \hline
		\multicolumn{1}{|c|}{\textbf{D}} & 70         & 3E         & B5         & 66         & 48         & 03         & F6         & 0E         & 61         & 35         & 57         & B9         & 86         & C1         & 1D         & 9E         \\ \hline
		\multicolumn{1}{|c|}{\textbf{E}} & E1         & F8         & 98         & 11         & 69         & D9         & 8E         & 94         & 9B         & 1E         & 87         & E9         & CE         & 55         & 28         & DF         \\ \hline
		\multicolumn{1}{|c|}{\textbf{F}} & 8C         & A1         & 89         & 0D         & BF         & E6         & 42         & 68         & 41         & 99         & 2D         & 0F         & B0         & 54         & BB         & 16         \\ \hline
	\end{tabular}
\end{table}
Cada byte do vetor State é então atualizado para um valor correspondente contido na Caixa-S. Os quatro primeiros bits de um byte serão correspondentes a linha e os quatro últimos bits correspondem a coluna, com os quatro bits transformados em hexadecimal temos uma posição na Caixa-S, o elemento substituirá o atual no vetor State.
Para descriptografia é feito o mesmo processo, porém com a Caixa-S inversa.


\subsubsection{Estágio ShiftRows}
Neste estágio ocorre um simples deslocamento nas linhas. Na primeira linha não ocorre deslocamento, na segunda ocorre o deslocamento circular de 1 byte à esquerda, na terceira ocorre o deslocamento circular de 2 bytes à esquerda e na quarta ocorre o deslocamento circular de 3 bytes à esquerda. Como mostrado no exemplo abaixo:\par
\begin{table}[tbh]
	\centering
	\caption{Exemplo de ShiftRows}
	\label{tab3}
	\begin{tabular}{|l|l|l|l|ll|l|l|l|l|}
		\cline{1-4} \cline{7-10}
		87 & F2 & 4D & 97 & \textrightarrow &  & 87 & F2 & 4D & 97 \\ \cline{1-4} \cline{7-10} 
		EC & 6E & 4C & 90 & \textrightarrow &  & 6E & 4C & 90 & EC \\ \cline{1-4} \cline{7-10} 
		4A & C3 & 46 & E7 & \textrightarrow &  & 46 & E7 & 4A & C3 \\ \cline{1-4} \cline{7-10} 
		8C & D8 & 95 & A6 & \textrightarrow &  & A6 & 8C & D8 & 95 \\ \cline{1-4} \cline{7-10} 
	\end{tabular}
\end{table}
Para a decriptografia o deslocamento é feito para a direita, nas três últimas linhas.

\subsubsection{Estágio MixColumms}
Neste estágio ocorre atualização na matriz State, através de uma operação que relaciona cada coluna separadamente. Podemos expressar como a seguinte multiplicação de matrizes:
$$
 \left[
\begin{array}{cccc}
02  & 03 & 01 & 01  \\
01  & 02 & 03 & 01 \\
01  & 01 & 02 & 03 \\
03  & 01 & 01 & 02  \\
\end{array}
\right]
 \left[
\begin{array}{cccc}
S_{0,0}  & S_{0,1} & S_{0,2} & S_{0,3}  \\
S_{1,0}  & S_{1,1} & S_{1,2} & S_{1,3}  \\
S_{2,0}  & S_{2,1} & S_{2,2} & S_{2,3}  \\
S_{3,0}  & S_{3,1} & S_{3,2} & S_{3,3}  \\
\end{array}
\right]=
 \left[
\begin{array}{cccc}
S'_{0,0}  & S'_{0,1} & S'_{0,2} & S'_{0,3}  \\
S'_{1,0}  & S'_{1,1} & S'_{1,2} & S'_{1,3}  \\
S'_{2,0}  & S'_{2,1} & S'_{2,2} & S'_{2,3}  \\
S'_{3,0}  & S'_{3,1} & S'_{3,2} & S'_{3,3}  \\
\end{array}
\right]
$$
Os novos elementos para cada coluna serão então definidos pelas equações:
$$
\begin{array}{c}

S'_{0,j}=(2\cdot S_{0,j})\oplus(3\cdot S_{1,j})\oplus S_{2,j}\oplus S_{3,j} \\
S'_{1,j}=S_{0,j}\oplus(2\cdot S_{1,j})\oplus(3\cdot S_{2,j})\oplus S_{3,j} \\
S'_{2,j}=S_{0,j}\oplus S_{1,j}\oplus(2\cdot S_{2,j})\oplus (3\cdot S_{3,j}) \\
S'_{3,j}=(3\cdot S_{0,j})\oplus\cdot S_{1,j}\oplus S_{2,j}\oplus(2\cdot S_{3,j}) \\
\end{array}
$$

\subsubsection{Estágio AddRoundKey}
Nesse estágio e que incluímos a presença da chave no algoritmo. Cada bit da matriz State passa por um XOR bit a bit com a chave da rodada. O resultado desta operação é o que atualiza a matriz State. 
Para decriptografar esse estágio é igual, já que a operação XOR é a própria inversa.

\subsubsection{Algoritmo de Expansão de Chave}

Este algoritmo é utilizado para garantir que cada rodada tenha uma nova chave. Ele utiliza como entrada uma chave de 4 words e produz um vetor de 44 words. Garantindo uma chave de 4 words para cada rodada.
