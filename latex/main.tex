\documentclass[
% -- opções da classe memoir --
12pt,				% tamanho da fonte
openright,			% capítulos começam em pág ímpar
oneside,			% para impressão em pagina unica
a4paper,			% tamanho do papel. 
english,			% idioma adicional para hifenização
french,				% idioma adicional para hifenização
spanish,			% idioma adicional para hifenização
brazil,				% o último idioma é o principal do documento
]{abntex2}

% ---
% Pacotes fundamentais 
% ---
\usepackage{lmodern}			% Usa a fonte Latin Modern
\usepackage[T1]{fontenc}		% Selecao de codigos de fonte.
\usepackage[utf8]{inputenc}		% Codificacao do documento (conversão automática dos acentos)
\usepackage{indentfirst}		% Indenta o primeiro parágrafo
\usepackage{color}				% Controle das cores
\usepackage{graphicx}			% Inclusão de gráficos
\usepackage{microtype} 			% para melhorias de justificação
\usepackage{placeins}
\usepackage{bigfoot} % to allow verbatim in footnote
\usepackage[numbered,framed]{matlab-prettifier}
% ---
% Pacotes adicionais, usados no anexo do modelo de folha de identificação
% ---
\usepackage{multicol}
\usepackage{multirow}
% ---

% ---
% Pacotes adicionais, usados apenas no âmbito do Modelo Canônico do abnteX2
% ---
\usepackage{lipsum}				% para geração de dummy text
% ---

% ---
% Pacotes de citações
% ---
\usepackage[brazilian,hyperpageref]{backref}	 % Paginas com as citações na bibl
\usepackage[alf]{abntex2cite}	% Citações padrão ABNT

% --- 
% CONFIGURAÇÕES DE PACOTES
% --- 

% ---
% Configurações do pacote backref
% Usado sem a opção hyperpageref de backref
\renewcommand{\backrefpagesname}{Citado na(s) página(s):~}
% Texto padrão antes do número das páginas
\renewcommand{\backref}{}
% Define os textos da citação
\renewcommand*{\backrefalt}[4]{
	\ifcase #1 %
	Nenhuma citação no texto.%
	\or
	Citado na página #2.%
	\else
	Citado #1 vezes nas páginas #2.%
	\fi}%
% ---
%redefine a capa
\renewcommand{\imprimircapa}{%
	\begin{capa}%
		\center
		\ABNTEXchapterfont\Large \textbf{UNIVERSIDADE FEDERAL DE SERGIPE}
		\\
		\vspace*{1cm}
		{\ABNTEXchapterfont\large\imprimirautor}
		\vfill
		\begin{center}
			\ABNTEXchapterfont\bfseries\LARGE\imprimirtitulo
		\end{center}
		\vfill
		\large\imprimirlocal \\
		\large\imprimirdata
		\vspace*{1cm}
	\end{capa}
}
% ---
%----
% Informações de dados para CAPA e FOLHA DE ROSTO
% ---
\titulo{Encriptação AES}
\autor{Iuri Rodrigo Ferreira Alves da silva\\Gregory Medeiros Melgaço Pereira\\Raul Rodrigo Silva de Andrade \\ Rafael Castro Nunes \\ Ruan Robert Bispo dos Santos \\ Vítor do Bomfim Almeida Carvalho}
\local{São Cristóvão,SE}
\data{\today}
\instituicao{%
	Universidade Federal De Sergipe
	\par
	Faculdade de Engenharia Eletrônica
	\par
	Redes e Comunicações}
\tipotrabalho{Relatório técnico}
% O preambulo deve conter o tipo do trabalho, o objetivo, 
% o nome da instituição e a área de concentração 
\preambulo{Relatório em conformidade com as normas ABNT}
% ---

% ---
% Configurações de aparência do PDF final

% alterando o aspecto da cor azul
\definecolor{blue}{RGB}{41,5,195}

% informações do PDF
\makeatletter
\hypersetup{
	%pagebackref=true,
	pdftitle={\@title}, 
	pdfauthor={\@author},
	pdfsubject={\imprimirpreambulo},
	pdfcreator={LaTeX with abnTeX2},
	pdfkeywords={abnt}{latex}{abntex}{abntex2}{relatório técnico}, 
	colorlinks=true,       		% false: boxed links; true: colored links
	linkcolor=blue,          	% color of internal links
	citecolor=blue,        		% color of links to bibliography
	filecolor=magenta,      		% color of file links
	urlcolor=blue,
	bookmarksdepth=4
}
\makeatother
% --- 

% --- 
% Espaçamentos entre linhas e parágrafos 
% --- 

% O tamanho do parágrafo é dado por:
\setlength{\parindent}{1.3cm}

% Controle do espaçamento entre um parágrafo e outro:
\setlength{\parskip}{0.2cm}  % tente também \onelineskip

% ---
% compila o indice
% ---
\makeindex
% ---

% ----
% Início do documento
% ----
\begin{document}
	\lstset{language=Matlab} 
	% Seleciona o idioma do documento (conforme pacotes do babel)
	%\selectlanguage{english}
	\selectlanguage{brazil}
	
	% Retira espaço extra obsoleto entre as frases.
	\frenchspacing 
	
	% ----------------------------------------------------------
	% ELEMENTOS PRÉ-TEXTUAIS
	% ----------------------------------------------------------
	
	% ---
	% Capa
	% ---
	\imprimircapa
	% ---
	% Folha de rosto
	% (o * indica que haverá a ficha bibliográfica)
	% ---
	\imprimirfolhaderosto*
	% ---
	
	% RESUMO
	

\setlength{\absparsep}{18pt} %espaçamento dos parágrafos do resumo
	\begin{resumo}
	
	
		\noindent
		\textbf{Palavras-chaves}: Grafos, Problema dos Menores Caminhos, Floyd-Warshall, Dijkstra, Iterações
	\end{resumo}

	% ---
	
	%lista de ilustrações
	\pdfbookmark[0]{\listfigurename}{lof}
	\listoffigures*
	\pagebreak
	% ---

	%lista de tabelas
	\pdfbookmark[0]{\listtablename}{lot}
	\listoftables*
	\pagebreak
	% ---
	
	% sumario
	\pdfbookmark[0]{\contentsname}{toc}
	\tableofcontents*
	\pagebreak
	% ---
	
	\chapter{Introdução}

No contexto capitalista e competitivo atual é cada vez mais prescindível que os dados enviados e recebidos, principalmente online, sejam protegidos, em que apenas quem envia e quem recebe tenha acesso ao seu conteúdo, garantindo assim o direito de privacidade. Essa ideia de segurança de dados se aplica diretamente à diversas áreas como : troca de mensagens entre usuários de aplicativos, compras e processos financeiros online e troca de informações entre países ou entre organizações de um único país, já que muitos conteúdos são confidenciais e apenas autoridades do governo podem ter acesso.  Os fundamentos de segurança (REFERENCIAR) são definidos por disponibilidade, integridade, controle de acesso, autenticidade, não-repudiação e privacidade. Foi pensando-se nisso que a criptografia foi criada. A palavra criptografia que provém dos radicais gregos kriptos (oculto) e grafo (escrita), é o nome dado à ciência de codificar mensagens utilizando algoritmos que serão usados novamente para decodificar essa mensagem. A criptografia apresenta dois tipos básicos: Simétrica (chave fechada) e Assimétrica (chave aberta).

A criptografia assimétrica foi criada na década de 1970. Nesse modelo, cada dispositivo envolvido na comunicação possui dois tipos de chaves diferentes, uma particular e uma pública. Essas chaves são processos digitais complexos que podem eventualmente estar associados a senhas. A chave pública é conhecida por qualquer usuário e é utilizada quando se quer se comunicar com outro usuário de modo seguro. Já a chave particular apenas cada dispositivo conhece e tem a sua. É com essa chave particular que o destinatário pode descriptografar a mensagem que foi criptografada com sua respectiva chave pública. A mensagem pode ser entendido com um bem precioso, a chave pública o cadeado que protege esse bem e a chave particular é chave física capaz de abrir esse cadeado. A vantagem desse método é a segurança, já que não é necessário o compartilhamento das chaves particulares e elas se encontram em poder do destinatário e da fonte, não há risco de intercepção por terceiros para saber essa chave particular, eles apenas podem conhecer a chave pública do destinatário. É importante ressaltar que para um dispositivo enviar uma mensagem a outro, ele já tem que conhecer a chave pública do destino.  A desvantagem é que com esse método o tempo de processamento de mensagens fica muito maior que a criptografia simétrica. Vários algoritmos para a criptografia assimétrica já existem, como o RSA e o Elgamal. O algoritmo RSA é o algoritmo de chave pública mais amplamente utilizado, além de ser uma das mais poderosas formas de criptografia de chave pública conhecidas até o momento. O RSA utiliza números primos. A premissa por trás do RSA consiste na facilidade de multiplicar dois números primos para obter um terceiro número, mas muito difícil de recuperar os dois primos a partir daquele terceiro número

A criptografia simétrica é o modelo mais antigo de criptografia. Nesse modelo, a chave que dá acesso ao conteúdo da mensagem trocada entre dois dispositivos deve permanecer em segredo. Geralmente essa chave é representada por uma senha que é usada pelo remetente para codificar a mensagem e usada pelo destinatário para decodificar a mensagem. A vantagem desse modelo é a sua simplicidade. Caso a chave seja complexa o algoritmo não necessariamente precisa também ser muito complexo, o que é bom, já que quanto mais simples o algoritmo, maior é a sua velocidade de processamento e facilidade de implementação. A principal desvantagem deste modelo é que como é utilizada apenas uma chave para ciframento e desciframento, conhecendo a chave se tem acesso aos dois processos, o que pode ocorrer interceptando o canal utilizado. Com isso, é muito importante a comunicação por um canal seguro evitando assim a ação de intrusos que podem ter acesso a mensagem. Outros problemas desse método é que como cada par necessita de uma chave, em uma rede com ‘n’ usuários, serão necessárias $n^2$ chaves, o que dificulta o gerenciamento. Além disso, não é fácil armazenar essas chaves de forma segura. Com isso, esse método não garante os princípios de autenticidade e não-repudiação. Vários algoritmos para a criptografia simétrica já existem, como o AES e o DES. O algoritmo AES é o mais utilizado e é o adotado como padrão pelo governo dos EUA. Esse algoritmo possui um bloco fixo em 128 bits e uma chave com tamanho de 128, 192 ou 256 bits, é relativamente fácil de executar e requer pouca memória. Esse algoritmo será o utilizado neste projeto.
	\phantompart
	\chapter{Referência}
	\phantompart
	\chapter{Objetivos}
	\phantompart
	\chapter{Formulação do problema}
	\phantompart
	\chapter{Resultados Obtidos}
	\phantompart
	\chapter{Conclusão}
	\phantompart
	
	
	%%%%%%%%%%%%%%%%%%%%%%%%%%%%%%%%%%%%%%%%%
	%Referência
	%%%%%%%%%%%%%%%%%%%%%%%%%%%%%%%%%%%%%%%%%
	\nocite{oliveira,dacriptografia}
	\bibliography{biblio}
	
	
	%%%%%%%%%%%%%%%%%%%%%%%%%%%%%%%%%%%%%%%%%
	%the end
	%%%%%%%%%%%%%%%%%%%%%%%%%%%%%%%%%%%%%%%%%
\end{document}
